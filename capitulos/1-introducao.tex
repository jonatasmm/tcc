\chapter{Introdução}

\section{Problemas}
Um dos grandes problemas encontrados nos softwares livres é a baixa usabilidade de suas interfaces, o que resulta na perda de usuários. 
Os desenvolvedores de software livre possuem uma mentalidade mais voltada para a funcionalidade do que para os usuários do sistema [Thor08]. Possuem código de qualidade, com algarismos eficientes e de bom desempenho e são produzidos por desenvolvedores motivados e voluntários. [Ana Paula]

\section{Objetivos}

\subsection{Objetivos Gerais}

O objetivo geral deste trabalho é analisar quais são as técnicas e métodos de usabilidade que podem ser inseridas no ciclo de vida do desenvolvimento ágil de software livre. 

\subsection{Objetivos Específicos}

Como objetivos específicos este trabalho visa:

\begin{enumerate}

\item Utilizar as práticas de usabilidade que podem ser inseridas ao longo do processo  ágil de desenvolvimento de software.
\item Analisar a usabilidade dos portais governamentais utilizando diferentes técnicas de avaliação, a fim de que a avaliação possa ser feita não somente pelo especialista de usabilidade, mas também para que os usuários avaliem a qualidade em uso dos portais.
\item Propor uma arquitetura de informação que auxilia na participação social.

\end{enumerate}


\section{Metodologia}

\subsection{Classificação da Pesquisa}

Neste trabalho, a coleta e análise dos dados serão realizadas com base em  materiais já publicados, constituído principalmente de livros, artigos de periódicos e  materiais disponibilizados na Internet (GIL, 1991); caracterizando-se, portanto, como uma pesquisa bibliográfica do ponto de vista do procedimento técnico empregado. 
Do ponto de vista da natureza, a pesquisa é aplicada pois tem como objetivo gerar conhecimentos para aplicação prática, drigida à solução de problemas específicos. Do ponto de vista da forma de abordagem do problema será tanto qualitativa como quantitativa.

\subsection{As etapas da Pesquisa}

\subsubsection{Estudo bibliográfico}

	Levando em consideração os objetivos da pesquisa, o estudo bibliográfico aborda a usabilidade e a sua relação com as diferentes áreas de conhecimento, tentando entender como a usabilidade está presente em nosso meio. As áreas pesquisadas são (Arquitetura da Informação, Design, Ergonomia, Interação Humano Computador, Engenharia de Usabilidade, Experiência do Usuário, Psicologia, entre outras). Deve ser feito um estudo sobre o processo  de desenvolvimento de software livre. Além disso é necessário conhecer o ciclo de design centrado no usuário que é a base de toda a pesquisa. Encontrar maneiras de como inserir os métodos de usabilidade dentro do contexto de software livre. 
	Pesquisa dos principais paradigmas de avaliação de usabilidade, descrevendo as técnicas utilizadas e sua aplicação para a engenharia de software.

\subsubsection{Coletas de dados}

	Os dados serão coletados através de questionários e entrevistas que serão feitas através de experimentos realizados em um estudo de caso no portal da participação social.



