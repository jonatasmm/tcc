\chapter{Usabilidade}

\section{As áreas da Usabilidade}

	Para entender o que é usabilidade e como ela está inserida no ciclo de vida do desenvolvimento de software precisamos compreender as relações que o termo tem com as diversas áreas que a envolve.
	A usabilidade é um termo antigo que começou a ser utilizado pela Ciência Cognitiva e depois pela Psicologia e Ergonomia em substituição ao termo “amigável”. (Cláudia Dias, 2007). 

\subsection{Interação Humano Computador}

	O Termo Human Computer Interaction  (HCI) começou a ser adotado na década de 1980 para descrever uma nova área de estudo na qual se preocupavam em saber como o uso dos computadores poderia enriquecer a vida pessoal e profissional dos seus usuários. (Moraes, 2002). 
	A interação humano computador tem o principal objetivo de melhorar a eficácia e proporcionar satisfação do usuário. Para Preece (1994), o objetivo do HCI são desenvolver e aprimorar sistemas computacionais nos quais os usuários possam executar tarefas com segurança, eficiência e satisfação.

	
\subsection{Arquitetura da Informação}

A informação é algo que está presente no nosso dia-a-dia. Para (Wurman, 1991) a informação deve ser aquilo que leva à compreensão. A quantidade de conteúdo que é produzido na internet extrapola a capacidade humana de retenção da informação. (Santa Rosa). Esse excesso de informação contribui para o aumento dos problemas de usabilidade e da necessidade de pesquisa na área de interação humano-computador. (Agner, 2004).
Segundo Garrett (2003), a arquitetura da informação são a arte e a ciência de estruturar e organizar os ambientes informacionais para ajudar as pessoas a encontrarem e administrarem informações.
O arquiteto de informação deve ser um profissional multidisciplinar com conhecimentos em design gráfico, ciência da informação, biblioteconomia, jornalismo, engenharia de usabilidade, marketing e ciência da computação. Ele deve balancear as necessidades do usuário com os objetivos do negócio. (Rosenfeld e Morville, 1998).


\subsection{Experiência do Usuário - UX}
É toda a intereação que temos com um produto, serviço ou marca.
UX é um termo usado frequentemente para sintetizar toda a experiência com um produto de software. Não engloba  somente nas funcionalidades (Travis)

%Verificar as leituras

\subsection{Relação de todas essas áreas com a Usabilidade}

\section{A importância da Usabilidade}

O que impede a implatação dos testes de usabilidade nas equipes de software?
Tempo e recursos limitados impediam de realizar testes de usabilidade

ROI - Retorno do Investimento

O porquê muitas vezes não são investidos recursos e esforços em método e técnicas de usabilidade.

Incorporar a usabilidade no seu processo pode reduzir os custos e tempo de desenvolvimento e melhorar o produto final. Ter em mente  em quem é o usuário final em todas as etapas de desenvolvimento e processos de produção, desde análise das necessidades e projeto conceitual até prototipagem e produção. 

%pg 33 (Avaliação de projetos de design de interface)

Benefícios da usabilidade
Leal
Borset 
outros
% falta escrever sobre o assunto.


\section{Design Centrado no usuário}

É um processo de criação que se baseia nas necessidades, desejos e limitações das pessoas. O usuário deve está presente no ínicio ao fim do projeto.
O DCU deve-se iniciar com usuários e suas necessidades em vez de começar com a tecnologia.
O DCU surgiu da IHC e consiste em uma metodologia de design de software para 

%	Detalhar o fluxo do DCU

Para criar produtos que os usuário amem, é necessário incluir os usuários no processo de criação dos produtos (Travis Lowrdemilk)
