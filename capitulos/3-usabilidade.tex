\chapter{Usabilidade}

\section{As áreas da Usabilidade}

	Para entender o que é usabilidade e como ela está inserida no ciclo de vida do desenvolvimento de software precisamos compreender as relações que o termo tem com as diversas áreas que a envolve.
	A usabilidade é um termo antigo que começou a ser utilizado pela Ciência Cognitiva e depois pela Psicologia e Ergonomia em substituição ao termo “amigável”. (Cláudia Dias, 2007). 

	Segundo a ISO 9241, usabilidade é a capacidade de um produto ser usado por usuários específicos para alcançar objetivos específicos com eficácia, eficiência e satisfação em um contexto de uso específico.
	A usabilidade é a qualidade que caracteriza o uso dos programas e aplicações. Não é uma qualidade intríseca de um sistema...
	
	De acordo com a ISO 9126-4, qualidade em uso é a capacidade do produto de software para permitir que usuários específicos atinjam objetivos específicos com eficiência, produtividade, segurança e satisfação em um contexto especifico [ISO 9126]. A qualidade em uso engloba o contexto do ambiente de trabalho para caracterizar a satisfação de uso, focando não apenas no usuário, mas em seu comportamento ao interagir com um sistema computacional. O principal objetivo é a avaliação de como as características do software atendem às necessidades dos usuários. Esta qualidade é analisada sob quatro características: eficiência, produtividade, segurança e satisfação.


\subsection{Interação Humano Computador}

	O Termo Human Computer Interaction  (HCI) começou a ser adotado na década de 1980 para descrever uma nova área de estudo na qual se preocupavam em saber como o uso dos computadores poderia enriquecer a vida pessoal e profissional dos seus usuários. (Moraes, 2002). 
	A interação humano computador tem o principal objetivo de melhorar a eficácia e proporcionar satisfação do usuário. Para Preece (1994), o objetivo do HCI são desenvolver e aprimorar sistemas computacionais nos quais os usuários possam executar tarefas com segurança, eficiência e satisfação.

	
\subsection{Arquitetura da Informação}

A informação é algo que está presente no nosso dia-a-dia. Para (Wurman, 1991) a informação deve ser aquilo que leva à compreensão. A quantidade de conteúdo que é produzido na internet extrapola a capacidade humana de retenção da informação. (Santa Rosa). Esse excesso de informação contribui para o aumento dos problemas de usabilidade e da necessidade de pesquisa na área de interação humano-computador. (Agner, 2004).
Segundo Garrett (2003), a arquitetura da informação são a arte e a ciência de estruturar e organizar os ambientes informacionais para ajudar as pessoas a encontrarem e administrarem informações.
O arquiteto de informação deve ser um profissional multidisciplinar com conhecimentos em design gráfico, ciência da informação, biblioteconomia, jornalismo, engenharia de usabilidade, marketing e ciência da computação. Ele deve balancear as necessidades do usuário com os objetivos do negócio. (Rosenfeld e Morville, 1998).

\subsection{Ergonomia}

A ergonomia está na origem da usabilidade, pois visa proporcionar eficácia e eficiência, além de bem-estar e saúde do usuário através da adaptação do trabalho ao homem. O seu objetivo é garantir que sistemas e dispositivos estejam adaptados às maneiras de pensar, agir e trabalhar do usuário. (Cybis) 


\subsection{Engenharia de Usabilidade}

A engenharia de usabilidade surgiu no final dos anos 80 com o esforço sistemático das empresas e organizações para desenvolver programas de software interativo com usabilidade. Sua origem parte de iniciativas de cientistas como Card, Moran e Newell (Modelo do processador Humano de 1983) e Norman (Teoria da Ação de 1989). O objetivo era produzir conhecimentos que favorecesse a concepção de interfaces humano-computador mais adaptadas. (Cybis)
%
Podemos fazer um paralelo da engenharia de usabilidade com a engenharia de software. A engenharia de software ocupa do desenvolvimento do núcleo funcional de um sistema interativo formado por estruturas de dados, algoritmos e recursos de processamento de dados. Esse núcleo é construído de forma que o sistema funcione bem, de forma correta, rápida e sem erros. Já a engenharia de usabilidade ocupa-se da interface com o usuário que interliga as funções do sistema com os usuários de forma que a interface do sistema seja agradável, intuitivo, eficiente e fácil de operar. (Cybis)

\subsubsection{Ciclo de Engenharia de Usabilidade}

	O ciclo foi definido sendo essencialmente evolutivo, interativo e baseado no envolvimento do usuário. A norma ISO 13407 (Projeto centrado no usuário) sugere quatro principais etapas desse ciclo (analisar e especificar o contexto de operação; especificar as exigências dos usuários e organizações; produzir soluções de projeto; avaliar o projeto contra as exigências). (Cybis)

Colocar figura: Projeto centrado no usuário

\subsection{Experiência do Usuário - UX}
É toda a interação que temos com um produto, serviço ou marca.
UX é um termo usado frequentemente para sintetizar toda a experiência com um produto de software. Não engloba  somente nas funcionalidades (Travis)

%Verificar as leituras

\subsection{Relação de todas essas áreas com a Usabilidade}

\section{A importância e os benefícios da Usabilidade}

No modo geral, os projetistas sabem da importância de desenvolver com enfoque no usuário e na usabilidade, mas normalmente os projetos são desenvolvidos sem que tenham sido realizadas pesquisas e aplicados métodos e técnicas de usabilidade.
	
	O tempo e os recursos limitados são as principais razões que impede a implatação dos testes de usabilidade nas equipes de software. Também há o desconhecimento por parte da equipe de desenvolvimento das técnicas a serem empregadas.

Incorporar a usabilidade no seu processo pode reduzir os custos e tempo de desenvolvimento e melhorar o produto final. Ter em mente  em quem é o usuário final em todas as etapas de desenvolvimento e processos de produção, desde análise das necessidades e projeto conceitual até prototipagem e produção.  (reescrever)

	O investimento na área de usabilidade agrega valor ao produto e traz beneficios não somente aos usuários, mas também aos seus produtores. 

Para a Associação de Profissionais de usabilidade (UPA), a usabilidade inclui os seguintes benefícios:

\begin{itemize}
\item Aumentar a produtividade
\item Diminuir custos de treinamento e suporte
\item Aumentar as vendas e as revendas
\item Reduzir os custos de desenvolvimento e manutenção.
\item Aumentar a satisfação do consumidor.
\end{itemize}


\section{Design Centrado no usuário}

É um processo de criação que se baseia nas necessidades, desejos e limitações das pessoas. O usuário deve está presente no ínicio ao fim do projeto.
O DCU deve-se iniciar com usuários e suas necessidades em vez de começar com a tecnologia.
O DCU surgiu da IHC e consiste em uma metodologia de design de software para 

%	Detalhar o fluxo do DCU

Para criar produtos que os usuário amem, é necessário incluir os usuários no processo de criação dos produtos (Travis Lowrdemilk)

\section{Métodos Agéis e o DCU}

\subsection{O ciclo da engenharia de usabilidade e as abordagens para o desenvolvimento de software}

	Exitem 2 grandes abordagens para o desenvolvimento de software: A abordagem tradicional sendo o seu principal método o (RUP - Rational Unified Process) e a abordagem ágil.
	Nesse trabalho iremos abordar somente os métodos ágeis de desenvolvimento de software que é o principal método utilizado no desenvolvimento de software livre atualmente.

Explicar os métodos àgeis

Citar manifesto ágil

As características da abordagem ágil facilitam na utilização da ergonomia e da usabilidade durante o desenvolvimento de software, mas os ergonomistas e engenheiros de usabilidade deverão adaptar suas técnicas de análise, modelagem, projeto e teste adotando-se os preceitos do manifesto ágil. (Cybis)

\begin{itemize}
\item Modelagem e projeto de interfaces devem ser orientados a padrões de projeto.
\item As avaliações de ergonomia, testes de usabilidade e especificações de revisões de interface devem ser realizados rapidamente.
\end{itemize}

Citar a filosofia de trabalho que a  ISO 13407 define para as empresas que vise a usabilidade.
