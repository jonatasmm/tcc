\chapter{Técnicas e Métodos adotados em Usabilidade}


\section{Paradigmas de Avaliação}

Quatro cenários de aplicações dos testes caracterizam os cenários mais usados nos testes de avaliação: rápida e suja, teste de usabilidade, estudo de campo e avaliação preditiva. Dentre as técnicas de coleta de dados veremos como acontecem as observações, as solicitações de opinião, testar o desempenho dos usuários e modelar o desempenho das tarefas dos usuários a fim de prever a eficácia de uma interface. Estes cenários de testes são ainda apoiados por processos diferenciados de coleta de dados, sendo a observação e a solicitação de opinião os mais utilizados em cenários ou paradigmas com especialistas e usuários teste.

\subsection{Rápida e Suja}

A prática dessa avaliação consiste em obter um feedback informal dos usuários ou consultores, para confirmar que tudo que foi pedido está sendo colocado em prática, e que está sendo executado do modo que agrade os usuários e consultores. Essa avaliação pode ser feita em qualquer estágio de desenvolvimento, e sua ênfase está em uma contribuição rápida.

\subsection{Teste de Usabilidade}

É um dos métodos de teste de experiência do usuário (UX) mais frequentemente utilizado e conhecido entre aqueles que não são projetistas da UX.
Consiste em avaliar o desempenho dos usuários na execução de tarefas cuidadosamente preparadas, tarefas estas dentro do escopo do sistema. Esse desempenho pode ser avaliado no quesito, número de erros e tempo de execução da tarefa, questionários e entrevistas também podem ser utilizados.


\subsection{Estudo de Campo}

Estudo de campo é realizado no contexto real, com objetivo de aumentar o entendimento do que os usuários fazem naturalmente e de como a tecnologia impacta nessas atividades. Pode ser utilizado para ajudar a identificar oportunidade de uso de novas tecnologias, determinar requisitos de design, facilitar a introdução da nova tecnologia, e avaliar a tecnologia.

\subsection{Avaliação Preditiva}

Consiste na aplicação de conhecimento dos especialistas, geralmente guiado por heurísticas, visando prever problemas de usabilidade. Não é necessária a presença dos usuários, tornando o processo mais rápido, relativamente barato, e, portanto atrativo para as empresas.


\subsection{Comparativo dos paradigmas de avaliação}

\subsection{Relação entre os paradigmas e as técnicas de avaliação}


\section{QUESTIONÁRIOS DE SATISFAÇÃO}

A aplicação de questionários é um dos métodos mais utilizados para avaliação da satisfação do usuário. Eles resultam da avaliação subjetiva pelo usuário, o qual é influenciado pelos tipos de questões aplicadas.
Um grande número de questionários foram desenvolvidos pela comunidade científica para a avaliação da usabilidade.  Alguns exemplos de questionários são: QUIS, SUMI,  WAMMI, SUS, ASQ, PSQ,PSSUQ, CSUQ. 

\subsection{QUIS}

	Questionnaire for User Interaction Satisfaction (QUIS) - mede a satisfação do usuário quanto â usabilidade do produto de maneira padronizada, segura e válida, a fim de obter informações precisas em relação à reação dos usuários a novos produtos (QUIS, 2009);
	A versão atual é a QUIS 7.0 (Norman e Shneiderman, 2010), contém um questionário onde possui a avaliação da satisfação geral e avaliações de fatores específicos de interfaces organizadas hierarquicamente: tela, terminologia e retroalimentação do sistema, aprendizado, capacidades do sistema, manuais técnicos, tutoriais online, multimídia, teleconferência e instalação de software. Pode ser configurado de acordo com a necessidade e interesse do usuário. 
	É um questionário proprietário, é sugerido o uso de planilhas eletrônicas e softwares estatísticos até que se implementem recursos de análise no servidor web dos proprietários.
	
\subsection{SUS – SYSTEM USABILITY SCALE}

O SUS (Brooke, 2010) é uma escala de usabilidade do tipo Likert que possui uma visão global e subjetiva em suas avaliações de usabilidade. Ele apresenta ao entrevistado uma lista de perguntas que devem ser respondidas em uma escala de satisfação (indica o grau de concordância ou discordância do usuário). 
O autor se baseou na afirmação de que no contexto industrial, as avaliações completas não são práticas e requerem muito esforço e custo. O SUS foi criado pela necessidade de se ter uma avaliação de usabilidade simples e rápida. Os métodos de avaliação foram simplificados e o número de questões reduzidas, pois uma quantidade grande de questões desanima os usuários que possivelmente não preencheria todas as questões, resultando assim problemas na captura de reações subjetivas do usuário. Foi então proposto um questionário com 10 questões que utiliza a escala Likert de cinco ou sete pontos. Este questionário abrange vários aspectos da usabilidade, tais como: necessidade de suporte, treinamento e complexidade.

\subsection{SUMI – USABILITY MEASUREMENT INVENTORY}

	O SUMI (Kirakowski e Corbett, 1988) é um questionário para medição da qualidade de um software do ponto de vista do usuário, é um método consistente usado para avaliar a qualidade de uso de um produto de software ou protótipo, e pode ajudar na descoberta de falhas de usabilidade (SUMI, 2009); É mencionado na norma ISO 9241 como um método reconhecido para testar a satisfação do usuário. O SUMI é um questionário comercial. 
	Inicialmente continha 150 itens onde o participante escolhia se (concordo fortemente, concordo, não sei, discordo ou discordo totalmente). Atualmente são 50 itens divididos em 5 grupos de 10 itens. Os grupos de itens são: eficiência, afeto, eficácia, controle e aprendizado. Os entrevistados preenchem o questionário no seu local de trabalho e devem decidir entre as opções: concordo, não sei ou discordo totalmente.

\subsection{ASQ – THE AFTER-SCENARIO QUESTIONNAIRE}

O ASQ (Lewis, 1995) é um questionário de três itens que são utilizados para avaliar a satisfação do usuário após a conclusão de cada cenário/tarefa. São realizadas umas séries de tarefas que estão de acordo com a realidade do usuário.
Este questionário aborda questões como: facilidade de conclusão da tarefa, tempo para completar uma tarefa e adequação das informações de suporte. São questões do tipo Likert (McIver e Carmines, 1981; Nunnally, 1978). É aplicada uma escala de 1 a 7, onde 1 representa “Concordo” e 7 para “Discordo totalmente”. 
O participante gasta em média 1 hora pra realizar cada cenário, no fim de cada cenário é preenchido o questionário ASQ. Após completar todos os cenários, no fim de 1 dia de trabalho (8 horas), os participantes preenchem o questionário PSSUQ para avaliação geral do sistema.
O ASQ foi aplicado na IBM por diferentes tipos de usuários, cada grupo possuía um tempo de experiência com sistemas de computador, o que permitiu a análise psicométrica do questionário.


\subsection{PSQ – THE PRINTER-SCENARIO QUESTIONNAIRE}

O PSQ(Lewis, 1995)  é uma versão inicial do ASQ, mas difere no formato e numero de itens.  São escalas de 5 pontos com os termos “Aceitável” com nota 1 e “Precisa de muita Melhoria” com nota 5, e não marcado “Para Avaliar”.  

\subsection{PSSUQ – THE POST-STUDY SYSTEM QUESTIONNAIRE}

O PSSUQ((Lewis, 1995)) fornece uma avaliação global do sistema utilizado. Possui 19 itens para avaliação da satisfação do usuário com a usabilidade do sistema. É gasto em média 10 minutos para completar o questionário, mas só é preciso completar uma vez o questionário no fim do estudo de usabilidade. 
É aplicada uma escala de 7 pontos, onde o nível mais baixo  representa “Concordo” e o mais alto, 7, representa “Discordo Totalmente” e “Não aplicável” em um ponto fora da escala.
Este questionário ajuda a entender quais aspectos do sistema o usuário está mais preocupado. Ele avalia as características como facilidade de uso e de aprendizado, simplicidade, eficácia, informação e a interface com o usuário.
Existem 4 tipos de pontuações para as respostas aos itens do PSSUQ: Escore da satisfação geral (OVERALL), a utilidade do sistema(SYSUSE), a qualidade da  informação (INFOQUAL) e a qualidade da interface (INTERQUAL). 
GLOBAL – 0,97; SYUSE = 96; INFOQUAL = 91 E INTERQUAL = 91
A escala Global está relacionada com a soma das classificações ASQ que os participantes deram após completar cada cenário. 

\subsection{CSUQ}

Este questionário é parecido com o PSSUQ, mas a sua redação e diferente. Enquanto no PSSUQ afirma que “Eu poderia efetivamente realizar as tarefas e cenários usando este sistema” o CSUQ escreve: “Eu posso terminar meu trabalho de forma eficaz usando esse sistema?”. 
Na IBM, este questionário foi aplicado através de e-mail, enviado para funcionários de diferentes locais, o que houve uma maior quantidade de participantes, do que feito com grupos reduzidos presencialmente.
Utilize o CSUQ quando o estudo de usabilidade é em um ambiente fora do laboratório.
A confiabilidade foi de 0,93.

\subsection{Comparativo dos questionários}

Os questionários ASQ e PSQ são utilizados após a realização de um cenário. Contém os mesmos itens, mas possuem escalas diferentes. O ASQ possui uma maior confiabilidade em relação ao PSQ. 
PSSUQ e CSUQ são ambos os questionários de satisfação global. O PSSUQ utiliza itens adequados para uma situação de teste de usabilidade, já o CSUQ são apropriados para uma situação de teste de campo. Os questionários possuem propriedades psicométricas aceitáveis de usabilidade e podem ser usados com confiança como medidas padronizadas de satisfação. É interessante utilizar o PSSUQ junto com o ASQ.
O ideal é que o questionário seja mais genérico possível. Cada questionário possui um nível de confiança.


\section{Como testar a usabilidade?}

O teste de usabilidade é um dos métodos de teste de experiência do usuário (UX) mais frequentemente utilizado e conhecido entre aqueles que não são projetistas da UX.

Existe alguns passos comuns que devem ser seguidos para a execução dos testes de usabilidade:

Escolher abordagem;
Planejar a pesquisa;
Recutramento e logística;
Criação de guias de discussão;
Facilitação;
Análise e apresentação dos resultados;
Criação de recomendações.


\subsection{Escolhendo uma abordagem}

As abordagens de pesquisa podem ser de dois tipos: Quantitativa ou qualitativas. 
Pesquisas quantitativas são focadas nos dados numéricos e é voltada para fornecer alta confiança e resultados repetidos dentro de seus grupos de usuários.É preciso ter o envolvimento de um número maior de participantes para contar as variações que você encontrará de indivíduo para indíviduo.(UNGER, CHANDLER)

As pesquisas qualitativas não são focadas em níveis de segurança e da possibilidade de repetição, mas sim ganhar contexto e percepção considerando o comportamento do usuário.Ela depende da interpretação do projetista sobre as descobertas, a intuição e o senso comum.(UNGER, CHANDLER)

Para os testes de usabilidade é possível utilizar qualquer uma das abordagens, mas a qualitativa é a mais acessível para aqueles que não tiveram um treinamento em métodos científicos mais formais e oferece uma rica fonte de dados. (UNGER, CHANDLER)

%Importante explicar o que teria no teste de usabilidade para cada tipo de abordagem

\subsection {Usuários}

Existem algumas diretrizes que podem ser adotadas para a definição da quantidade de usuários. Jacob Nielsen definiu algumas dessas diretrizes:

No teste quantitativo planeje uma quantidade maior de participantes. Em média 20 por rodada de pesquisa.
No teste qualitativo é suficiente ter entre 5 e 8 participantes.


\subsection {Planejamento da Pesquisa}

Algumas questões devem ser respondidas ao criar o teste de usabilidade: Estas questões te ajudam a oferecer foco e escopo.Abaixo algumas perguntas que devem ser respondidas no planejamento de sua pesquisa:

Defina seu objetivo: Por que você está testando? %Cap2
Defina Usuários: Quem você está testando? %Cap6e7
Defina o método para representar sua aplicação: O que você está testando?
Quais informações você está reunindo? 


Nas pesquisas qualitativas geralmente queremos compreender as questões que os usuários podem encontrar, os níveis de frustações que eles podem experimentar e a gravidade de um problema em particular. Para os testes qualitativos devem se pensar em medidas que serão possíveis de ser respondidas com 5 usuários. 

Taxa de Sucesso: O grau em que o usuário foi capaz de completar a tarefa?
%Pode-se detalhar com mais informações sobre o que seria a taxa de sucesso, e o que é considerado o sucesso.
Satisfação do usuário

\subsection{Recrutamento}

Depois de ter feito o plano de pesquisa e definido os tipos de usuários que podem ser inseridos na pesquisa é hora de recrutar os participantes.
Para o recrutamento de participantes é interessante gerar uma lista com os potenciais participantes do teste. Essa lista pode vir de usuários registrados do site da empresa relacionada; informações de contatos do cliente; e-mails para conhecidos que tenha relação com o assunto do teste; requisições em pequenas pesquisas que pré-qualificam os participantes, e etc.

Você pode realizar uma filtragem com os participantes potenciais antes de seleciona-los. As perguntas do questionário de filtragem devem ser voltadas para:

Garantir que o participante seja um usuário das funções em que você está testando.
Determinar se ele se encaixa em um dos seus grupos de usuários.
Ajudar a ter uma boa mistura de participantes.

O questionário de perfil de usuário pode ser utilizado para realizar essa filtragem de participantes.

\subsection{Teste da Usabilidade vs.Teste de Aceitação do Usuário (UAT)}


\subsection{Guias de Discussão}

Nas instruções devem conter todas as informações específicas que o participante precisará para completar a tarefa ou tarefas que você está testando com sucesso.

Os seus materiais de teste deve incluir:

Formulário de consentimento para gravação de vídeo
Guia de discussão para o participante, com tarefas detalhadas e perguntas sobre a satisfação do usuário.
Questionários

%verificar o que mais precisa

\subsection{Analisar Resultados do Teste}













