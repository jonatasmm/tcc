\begin{resumo}



Este trabalho tem como objetivo utilizar práticas de usabilidade que podem ser inseridas ao longo do processo de desenvolvimento de software livre. O objetivo proposto pelo estudo é analisar a usabilidade dos portais governamentais utilizando diferentes técnicas de avaliação, a fim de que a avaliação possa ser feita não somente pelo especialista de UX (Experiência do Usuário - User Experience), mas também para que os usuários avaliassem a qualidade em uso de alguns portais governamentais como: Participa.br e Portal do Software Público Brasileiro e mais alguns que serão escolhidos ao longo da pesquisa. Em suma, a ideia é como ponto de vista de um Engenheiro de Front-end mostrar as melhores práticas do desenvolvimento interface que possam ser aplicados para a melhoria da usabilidade dos portais governamentais que utilizam diferentes ferramentas de gerenciamento de conteúdo.
 \vspace{\onelineskip}
    
 \noindent
 \textbf{Palavras-chaves}: software livre. usabilidade. experiência do usuário.
\end{resumo}
